\documentclass[letterpaper,final,12pt,reqno]{amsart}

\usepackage[total={6.3in,9.2in},top=1.1in,left=1.1in]{geometry}

\usepackage{bm}
\usepackage{empheq}
\usepackage[dvipsnames]{xcolor}
\usepackage{graphicx}
\usepackage{verbatim,fancyvrb}
\usepackage{tikz}

% hyperref should be the last package we load
\usepackage[pdftex,
colorlinks=true,
plainpages=false, % only if colorlinks=true
linkcolor=blue,   % only if colorlinks=true
citecolor=Red,   % only if colorlinks=true
urlcolor=black     % only if colorlinks=true
]{hyperref}

\renewcommand{\baselinestretch}{1.05}

\newcommand{\ddt}[1]{\ensuremath{\frac{\partial #1}{\partial t}}}
\newcommand{\ddx}[1]{\ensuremath{\frac{\partial #1}{\partial x}}}
\newcommand{\ddy}[1]{\ensuremath{\frac{\partial #1}{\partial y}}}
\newcommand{\pp}[2]{\ensuremath{\frac{\partial #1}{\partial #2}}}
\renewcommand{\t}[1]{\texttt{#1}}
\newcommand{\Matlab}{\textsc{Matlab}\xspace}
\newcommand{\eps}{\epsilon}
\newcommand{\RR}{\mathbb{R}}

\newcommand{\grad}{\nabla}
\newcommand{\Div}{\nabla\cdot}
\newcommand{\trace}{\operatorname{tr}}


\newcommand{\hbn}{\hat{\mathbf{n}}}

\newcommand{\bg}{\mathbf{g}}
\newcommand{\bn}{\mathbf{n}}
\newcommand{\bu}{\mathbf{u}}
\newcommand{\bv}{\mathbf{v}}
\newcommand{\bx}{\mathbf{x}}

\newcommand{\bX}{\mathbf{X}}



\begin{document}
\graphicspath{{figures/}}

\title{Evolving-surface glacier flow using Stokes dynamics}

\author{Ed Bueler}

\maketitle

\date{\today}

\thispagestyle{empty}
\bigskip

FIXME Main points:
\renewcommand{\labelenumi}{\arabic{enumi}.}
\begin{enumerate}
\item mesh coordinates are part of the model state; we solve for them implicitly over a time step $\Delta t$
\item initially time-stepping is by backward Euler implemented by hand; later perhaps firedrake\_ts and BDF2 or such
\item at the start of the time step the domain in the $(x,\zeta)$ plane occupied by ice is called $\Lambda$, and it has a mesh; these are the base domain and mesh
\item equations are for variables $\bu=\left<u,v\right>,p,z$; all of these are functions of $(x,\zeta) \in \Lambda$
\item ``new'' domain occupied by the ice will be the set
    $$\Omega = \{(x,z(x,\zeta)) \,\big|\, (x,\zeta) \in \Lambda\}$$
\item we rewrite the weak-form integrals using change of coordinates
    $$(x,\zeta) \mapsto (x,z)$$
\item the $x$ coordinates of the mesh do not change; only vertical mesh motion
\item the $z$ value at the ice surface solves the time-discretized surface kinematical equation re-written into $(x,\zeta$ coordinates; this is a Robin boundary condition for $z$
\item in the interior of $\Lambda$, the difference $r(x,\zeta)=z(x,\zeta)-\zeta$ solves the Laplace equation subject to boundary values as follows:
    \begin{itemize}
    \item Dirichlet on surface from surface kinematical equation
    \item zero Dirichlet on base and inflow
    \item Neuman on outflow
    \end{itemize}
\item numerical solver should check element orientation under change of coordinates 6; if the solver flips an element then it is bad; also check new element aspect ratio and (presumably) remesh if that is bad
\item the initial iterate for the (SNES-based) solver is clear: $\bu,p$ come from solution of previous time step, and $r$ starts at zero
\end{enumerate}
 

FIXME The content below is just a copy of the McCarthy \texttt{stokes/doc/} notes

First we state the Stokes model for ice flow, with glacier-suitable boundary conditions.  This exact solution in this case is useful both for verification purposes and as a source of boundary conditions for general cases.  We then derive the ``weak form'' of the Stokes problem.  A brief overview of finite element (FE) methods, based on such weak forms, follows.  Our particular FE method uses an unstructured mesh of triangular elements on any planar region to solves the Stokes problem by a stable ``mixed element'' method with distinct approximating spaces for velocity and pressure.  Then we describe a moving-mesh scheme to solve the surface kinematical equation.  Finally we demonstrate a numerical solution for the instantaneous velocity and pressure, and for an evolving glacier shape.


\section{Glen-Stokes model} \label{sec:stokes}

Recall the Glen-Stokes model in equations (3), (4), (5) from the notes; the model is also described in \cite{GreveBlatter2009,JouvetRappaz2011}.  It applies on a 2D or 3D domain $\Omega$ which must have a piecewise smooth boundary, so that we may apply the boundary conditions, but is otherwise general.

Allowing any Glen exponent $n\ge 1$, the equations are:
\begin{align}
- \nabla \cdot \tau + \nabla p &= \rho \bg &&\text{\emph{stress balance}} \label{forcebalance} \\
\nabla \cdot \bu &= 0 &&\text{\emph{incompressibility}} \label{incompressible} \\
D\bu &= A_n |\tau|^{n-1} \tau &&\text{\emph{Glen flow law}} \label{flowlaw}
\end{align}
The notation here generally follows Table 1 in the notes, including velocity $\bu$, pressure $p$, ice density $\rho$, acceleration of gravity $\bg$, deviatoric stress tensor $\tau$ and strain rate tensor $D\bu$.  Tensors $D\bu$ and $\tau$ are symmetric and have trace zero.  Recall that $D\bu$ is the symmetric part of the tensor $\grad \bu$, which differentiates velocity:
\begin{equation}
(D\bu)_{ij} = \frac{1}{2} \left(\grad\bu + \grad\bu^\top\right) = \frac{1}{2} \left((u_i)_{x_j} + (u_j)_{x_i}\right) \label{strainrate}
\end{equation}
The full (Cauchy) stress tensor $\sigma$ is the deviatoric stress tensor $\tau$ minus the pressure,
\begin{equation}
    \sigma = \tau - p\,I,  \label{cauchystress}
\end{equation}
so equation \eqref{forcebalance} simply says $-\Div \sigma = \rho \bg$.  One may derive from \eqref{cauchystress} that $p = -\frac{1}{3} \trace(\sigma)$ (in 3D), thus that the pressure is the negative of the average normal stress.  By definition $\Div\tau$ in \eqref{forcebalance} is a vector with components which are the divergences of the rows:
\begin{equation}
    \left(\nabla \cdot \tau\right)_i = \left(\tau_{i1}\right)_{x_1} + \left(\tau_{i2}\right)_{x_2} + \left(\tau_{i3}\right)_{x_3}  \label{divtaudefn}
\end{equation}
Note $\nabla\cdot \tau$, $\nabla p$, and $\bg$ are regarded as column vectors, and tensor norm notation is:
\begin{align*}
|\tau|^2 = \frac{1}{2} \trace\left(\tau^2\right) = \frac{1}{2} \tau_{ij} \tau_{ij}, \qquad |D\bu|^2 = \frac{1}{2} \trace\left((D\bu)^2\right) = \frac{1}{2} (D\bu)_{ij} (D\bu)_{ij}
\end{align*}

The viscosity form of \eqref{flowlaw} is equation (15) in the notes:
\begin{equation}
\tau = 2\nu D\bu = B_n |D\bu|^{\frac{1}{n} - 1} D\bu  \label{viscflowlaw}
\end{equation}
Here $B_n = (A_n)^{-1/n}$ is the $n$-dependent ice hardness in units $\text{Pa}\,\text{s}^{1/n}$.  From \eqref{viscflowlaw} we can eliminate $\tau$ from equation \eqref{forcebalance}, thereby rewriting the system in terms of velocity and pressure only:
\begin{align}
- \nabla \cdot \left(B_n |D\bu|^{\frac{1}{n} - 1} D\bu\right) + \nabla p &= \rho \mathbf{g} \label{stokes} \\
\Div \bu &= 0 \label{incompagain}
\end{align}
This system is the Glen-Stokes model, and a solution is a velocity-pressure pair $(\bu,p)$.

From now on we suppose the domain $\Omega$ is 2D, namely points denoted $(x,y,z)$ where $y=0$, and we denote the components of velocity as $\bu=\left<u,v,w\right>$.  We suppose there is no variation in the cross-flow direction ($\partial/\partial y=0$) and no cross-flow velocity ($v=0$).  Also we will assume the force of gravity is at an angle $\alpha$ with the $z$-direction so $\bg = \left<g\sin\alpha,0,-g\cos\alpha\right>$ where $g=|\bg|$.

Certain glacier-suitable velocity and stress boundary conditions will be used here.  We assume that the base, top, inflow, and outflow boundaries can all be identified.  On the base we require no slip:
\begin{align}
\bu &= 0  &&\text{\emph{base}} \label{basebc} \\
\intertext{On the top we set a condition of zero applied stress, i.e.~$\sigma\hbn=0$:}
\left(B_n |D\bu|^{\frac{1}{n} - 1} D\bu - pI\right) \hbn &= 0  &&\text{\emph{top}} \label{topbc} \\
\intertext{The left-side inflow boundary has outward normal $\hbn=\left<-1,0,0\right>^\top$ in our case, and on this surface we set a nonzero inflow velocity:}
\bu &= \left<f(z),0,0\right>^\top  &&\text{\emph{inflow}} \label{inflowbc} \\
\intertext{(The function $f(z)$ will satisfy the slab-on-slope equations for a specific thickness $H_{\text{in}}$ at the inflow; see below.)  On the outflow boundary, where $\hbn=\left<1,0,0\right>^\top$, we set a nonzero hydrostatic normal stress using the (varying) ice thickness $H_{\text{out}}$ at the outflow:}
\left(B_n |D\bu|^{\frac{1}{n} - 1} D\bu - pI\right) \hbn &= C_{\text{out}} \big<- \rho g \cos\alpha (H_{\text{out}}-z),0, &&\text{\emph{outflow}} \label{outflowbc} \\
    &\qquad\qquad \rho g\sin\alpha (H_{\text{out}}-z)\big>^\top  \notag
\end{align}
The constant $C_{\text{out}} $ is adjusted so that the total applied stress is equal to its value for the input ice thickness $H_{\text{in}}$, thus $C_{\text{out}} = (H_{\text{in}}/H_{\text{out}})^2$.


\section{Weak form} \label{sec:weakform}

The solution to the weak form is a pair $(\bu,p)$ with each function living in a certain function space.  Thus we write $\bu\in V_D$ and $p \in Q$, and function (Sobolev) spaces $V_D,Q$ are precisely-identified in \cite{JouvetRappaz2011}, but we ignore such abstraction in these notes.  Test functions $\bv\in V_0$ and $q\in Q$ come from nearly the same spaces.  The difference between $\bu$ and $\bv$ relates to the Dirichlet boundary conditions: $\bu\in V_D$ satisfies base and inflow boundary conditions \eqref{basebc} and \eqref{inflowbc}, respectively, while $\bv\in V_0$ is zero on those boundaries.

To give an initial definition of $F$ we multiply \eqref{stokes} by $\bv\in V_0$ and \eqref{incompagain} by $q\in Q$, then add and integrate:
\begin{equation}
F(\bu,p;\bv,q) = \int_\Omega - \left(\nabla \cdot \left(B_n |D\bu|^{\frac{1}{n} - 1} D\bu\right)\right)\cdot \bv + \nabla p \cdot \bv - \rho \mathbf{g} \cdot \bv - \left(\nabla \cdot \bu\right) q \label{nonfuncone}
\end{equation}
However, the final definition of $F$ will use integration-by-parts to rewrite $F$ so as to balance the number of derivatives on $(\bv,q)$ versus $(\bu,p)$.  For this step, recall the product rule $\nabla \cdot(f\bX) = \grad f\cdot \bX + f \nabla \cdot \bX$ and the divergence theorem $\int_\Omega \nabla \cdot \bX = \int_{\partial \Omega} \bX \cdot \hbn$.  Denoting $\tau = B_n |D\bu|^{\frac{1}{n} - 1} D\bu$ for typographical convenience we have
\begin{align*}
\int_\Omega \left(\nabla \cdot \tau\right)\cdot \bv &= \sum_{j=1}^3 \int_\Omega \nabla \cdot (\tau_{j\circ})\, v_j = \sum_{j=1}^3 \int_\Omega \nabla \cdot (\tau_{j\circ} v_j) - \tau_{j\circ} \nabla v_j \\
  &= \sum_{j=1}^3 \int_{\partial \Omega} (\tau_{j\circ} v_j) \cdot \hbn - \int_\Omega \tau_{j\circ} \cdot \nabla v_j = \int_{\partial \Omega} (\tau \hbn)\cdot \bv - \int_\Omega \trace(\tau \nabla \bv)
\end{align*}
where $\circ$ denotes a vector entry index and $\tau_{j\circ}$ denotes the $j$th row of $\tau$.  Here $\grad\bv$ defines a $3\times 3$ matrix,
\newcommand{\trefthree}[3]{\left[\begin{array}{c|c|c} & & \\ #1 & #2 & #3 \\ & & \end{array}\right]}
    $$\grad \bv = \trefthree{\grad v_1}{\grad v_2}{\grad v_3} = \begin{bmatrix}
    (v_1)_{x_1} & (v_2)_{x_1} & (v_3)_{x_1} \\
    (v_1)_{x_2} & (v_2)_{x_2} & (v_3)_{x_2} \\
    (v_1)_{x_3} & (v_2)_{x_3} & (v_3)_{x_3}
    \end{bmatrix}$$
and so
    $$\trace(\tau \grad \bv) = \sum_{j=1}^3 \tau_{j\circ} \cdot \grad v_j = \sum_{i,j=1}^3 \tau_{ji} (v_j)_{x_i}$$
(Some sources write $A:B$ for $\trace(AB)$ \cite{JouvetRappaz2011}.)  Note $\trace(\tau \grad \bv) = \trace(\tau D\bv)$ because $\trace(AB)=0$ if $A$ is symmetric and $B$ is antisymmetric.  (To show this take $A=\tau$ and $B=\grad\bv-D\bv$.)  Finally we do a straightforward integration-by-parts on the pressure part of $F$:
    $$\int_\Omega \nabla p \cdot \bv = \int_\Omega \nabla\cdot (p\,\bv) - p (\nabla \cdot \bv) = \int_{\partial \Omega} p\hbn \cdot \bv - \int_\Omega p (\nabla \cdot \bv)$$
The above facts allow us to rewrite \eqref{nonfuncone} with a normal stress boundary integral:
\begin{equation}
F(\bu,p;\bv,q) = -\int_{\partial\Omega} (\sigma \hbn)\cdot \bv + \int_\Omega \trace(\tau D\bv) - p (\nabla \cdot \bv) - \left(\nabla \cdot \bu\right) q - \rho \mathbf{g} \cdot \bv \label{nonfunctwo}
\end{equation}
(We have denoted $\sigma=\tau-pI$ for brevity.)  Now $\bu,\bv$ appear with at most first derivatives and $p,q$ appear without derivatives.  Next recall that $\bv\in V_0$ satisfies $\bv=0$ along the base and inflow surfaces.  Thus these parts of the integral over $\partial\Omega$ in \eqref{nonfunctwo} are zero.  Conditions \eqref{topbc}, \eqref{outflowbc} now completely eliminate the unknown solution $\bu,p$ from the boundary integral.

The above computations yield our final formula for the nonlinear functional:
\begin{align}
F(\bu,p;\bv,q) &= \int_\Omega B_n |D\bu|^{\frac{1}{n} - 1} \trace(D\bu D\bv) - p (\nabla \cdot \bv) - \left(\nabla \cdot \bu\right) q \label{defineF} \\
    &\qquad  - \int_\Omega \rho \mathbf{g} \cdot \bv - \int_{\{\text{outflow}\}} C_{\text{out}} \rho g \cos\alpha (h-z) v  \notag
\end{align}
The last two integrals can be regarded as source terms.  For example, if the inflow velocity is zero and if we replace the source terms by zero---no gravity or outflow stress---then the unique solution is $\bu=0$ and $p=0$.

In conclusion the weak form of the Glen-Stokes model is the statement that, at the solution $\bu\in V_D$ and $p\in Q$, functional $F$ in \eqref{defineF} is zero in all test function directions:
\begin{equation}
F(\bu,p;\bv,q) = 0 \qquad \text{ for all } \bv\in V_0 \text{ and } q\in Q  \label{weak}
\end{equation}
This weak formulation is proven in \cite[Theorem 3.8]{JouvetRappaz2011} to be well-posed under reasonable assumptions about the domain $\Omega$ and boundary data which are satisfied in the cases we consider.


\section{Surface kinematical equation} \label{sec:kinematical}

A glacier will change shape as it flows, and we want our model to include this action.  So now suppose that the domain on which the equations apply is the time-dependent set
\begin{equation}
\Omega^t = \left\{(x,z)\,\big|\, b(x) < z < h(x,t)\right\}  \label{Omegat}
\end{equation}
where $z=b(x)$ is the base elevation and $z=h(x,t)$ is the ice surface elevation.

The time-dependent surface $z=h(x,t)$ evolves according to the surface kinematical equation, (41) in the notes, using ice velocity $\bu=\left<u(x,z,t),w(x,z,t)\right>$:
\begin{equation}
h_t = a(x,t) - u(x,h,t) h_x + w(x,h,t) \label{surfacekinematical}
\end{equation}
Here $a(x,t)$ is the climatic mass balance in units of ice-equivalent $\text{m}\,\text{s}^{-1}$.  Informally, \eqref{surfacekinematical} determines the change in surface elevation $\Delta h \approx h_t\,\Delta t$ from the climatically added/removed ice $a\,\Delta t$ plus the component of the ice motion in the outward (upward) normal direction $\bn = \left<-h_x,1\right>$.  In summary, $\Delta h \approx \left(a + \bn\cdot \bu|_{z=h}\right) \Delta t$.

Equation \eqref{surfacekinematical} applies on the time-dependent ice surface $\Gamma^t = \left\{(x,z) \,\big|\, z = h(x,t)\right\}$.  The surface $\Gamma^t$ is also the $\Phi=0$ level surface of the function
    $$\Phi(x,z,t) = z - h(x,t)$$
\cite[pp.~65--66]{GreveBlatter2009}.  We regard $\Phi$ as being defined on the closure of $\Omega^t$, including the surface $\Gamma^t$.  Since the spatial gradient $\grad \Phi = \left<-h_x,1\right>$ is an outward normal along $\Gamma^t$ we have
\begin{equation}
h_t = a + \bu|_{\Gamma^t} \cdot (\grad \Phi)|_{\Gamma^t}  \label{surfacekinematicalwithPhi}
\end{equation}
In Firedrake the advantage of using \eqref{surfacekinematicalwithPhi}, over \eqref{surfacekinematical}, is that it is easier to compute the gradient of the 2D scalar field $\Phi(x,z,t_n)$, and evaluate it along the boundary, than it is to differentiate the surface elevation function $h(x,t_n)$.

In our examples below, because we are studying ice fluid dynamics in isolation, we set $a=0$.  Scientific questions generally require nontrivial models for $a$, but, given such a model, implementation of nonzero values for $a(x,t)$ is straightforward if applied in an explicit (time-split) manner.  Regarding the base, in our simplified model the elevation $b(x)$ is time-independent, and the base does not slide, nor melt/refreeze liquid water, and thus the base kinematical equation, (42) in the notes, reduces to ``$0=0$.''  Solving this equation requires no further effort.

We solve the surface kinematical equation \eqref{surfacekinematicalwithPhi} by smoothly and vertically displacing the mesh.  Note that the mesh is given by scalar coordinate fields $(x,z)$, defined on the nodes of the mesh, and that the time $t_n$ ice surface elevation $h^n(x) \approx h(x,t_n)$ equals the value of the coordinate field $z$ on the top boundary of the current mesh $\Omega^n$.  We make this surface elevation the boundary value of a Laplace equation problem for the vertical displacement $r^n$ of every node in the mesh.  In fact, we can now describe our explicit time-stepping strategy for moving the surface of a glacier.  Given the current geometry of the ice, i.e.~at time $t_n$, and a time step $\Delta t_n > 0$, one applies this sequence:

\medskip
\renewcommand{\labelenumi}{\emph{\arabic{enumi}.}}
\begin{enumerate}
\item Solve the weak-form Stokes equation \eqref{weak} on the current mesh $\Omega^n$ to compute current velocity and pressure values $(\bu^n,p^n)$.
\item From $\Omega^n$ also generate the piecewise-linear surface elevation function $h^n(x)$.  Then evaluate $\Phi^n(x,z) = z - h^n(x)$ on $\Omega^n$ and compute:
\begin{equation}
\Delta h^n(x,z) =  \left(a(x,t_n) + \grad \Phi^n(x,z)\cdot \bu^n(x,z)\right)\Delta t_n \label{deltahfield}
\end{equation}
\item Solve a Laplace equation problem for the vertical displacement field $r^n$:
\begin{align}
- \grad^2 r^n &= 0 & &\text{on } \Omega^n \label{vdisplacementpoisson} \\
          r^n &= \Delta h^n & &\text{(Dirichlet) on the top boundary } \Gamma^n \notag \\
          r^n &= 0 & &\text{(Dirichlet) on inflow and base of } \partial \Omega^n \notag \\
\grad r^n\cdot \bn &= 0 & &\text{(Neumann) on outflow of } \partial \Omega^n \notag
\end{align}
This linear problem is solved in weak form: $\int_{\Omega^n} \grad r^n\cdot \grad w = 0$ for all test functions $w$ with zero values on the Dirichlet part of the boundary $\partial \Omega^n$.
\item Update the mesh coordinates using vertical-only displacement by $r^n$:
\begin{equation}
  x^{n+1} = x^n, \quad z^{n+1} = z^n + r^n \label{updatemesh}
\end{equation}
\end{enumerate}

\medskip
At the end of this sequence the new top boundary $\Gamma^{n+1}$ is the surface $z=h^{n+1}(x)$.  By \eqref{deltahfield} the sequence has done an explicit step of \eqref{surfacekinematical}:
    $$h^{n+1}(x) = h^n(x) + \Delta t_n\,\left(a(x,t_n) - u(x,h^n,t_n) h_x^n + w(x,h^n,t_n)\right)$$

The significance of Dirichlet problem \eqref{vdisplacementpoisson} is that the entire mesh is \emph{smoothly} displaced, in the vertical direction only by \eqref{updatemesh}, because solutions of the Laplace equation are smooth.  (Note $r^n$ minimizes $\int |\grad f|^2$ over functions with the given boundary values.)  This way of displacing the mesh avoids shearing the mesh near sharp discontinuities in base topography.

Note that the scheme uses fixed time step $\Delta t > 0$ and is explicit.  At best it can be conditionally stable, and in practice that \emph{is} what is observed, but no time step restriction is known \emph{a priori}.  As with an SIA solver it might be expected that a sufficient condition like ``$D\Delta t / \Delta x^2 < 1$,'' where $\Delta x$ is a representative mesh spacing variable, perhaps taken near the surface and along the surface, and $D$ is some diffusivity parameter meaningful in the current context, might apply.  However, there is no literature which connects the $D$ from the SIA (see the notes) with Stokes time-stepping, or otherwise supplies this stability restriction.  Two ways of addressing this numerical modeling weakness have, I think, barely been started by the glacier modeling community:
\renewcommand{\labelenumi}{(\roman{enumi})}
\begin{enumerate}
\item extensively test various configurations to develop an empirical time-step restriction; it would depend on the size and aspect-ratio of the elements in some complicated way, or
\item solve the Stokes and surface kinematical equations, the later in implicit-step form, as a coupled system.
\end{enumerate}


\small

\bigskip
\bibliography{simp}
\bibliographystyle{siam}

\end{document}
