\documentclass[letterpaper,final,12pt,reqno]{amsart}

\usepackage[total={6.3in,9.2in},top=1.1in,left=1.1in]{geometry}

\usepackage{bm}
\usepackage{empheq}
\usepackage[dvipsnames]{xcolor}
\usepackage{graphicx}
\usepackage{verbatim,fancyvrb}
\usepackage{tikz}

% hyperref should be the last package we load
\usepackage[pdftex,
colorlinks=true,
plainpages=false, % only if colorlinks=true
linkcolor=blue,   % only if colorlinks=true
citecolor=Red,   % only if colorlinks=true
urlcolor=black     % only if colorlinks=true
]{hyperref}

\renewcommand{\baselinestretch}{1.05}

\newcommand{\ddt}[1]{\ensuremath{\frac{\partial #1}{\partial t}}}
\newcommand{\ddx}[1]{\ensuremath{\frac{\partial #1}{\partial x}}}
\newcommand{\ddy}[1]{\ensuremath{\frac{\partial #1}{\partial y}}}
\newcommand{\pp}[2]{\ensuremath{\frac{\partial #1}{\partial #2}}}
\renewcommand{\t}[1]{\texttt{#1}}
\newcommand{\Matlab}{\textsc{Matlab}\xspace}
\newcommand{\eps}{\epsilon}
\newcommand{\RR}{\mathbb{R}}

\newcommand{\grad}{\nabla}
\newcommand{\Div}{\nabla\cdot}
\newcommand{\trace}{\operatorname{tr}}


\newcommand{\hbn}{\hat{\mathbf{n}}}

\newcommand{\bg}{\mathbf{g}}
\newcommand{\bn}{\mathbf{n}}
\newcommand{\bu}{\mathbf{u}}
\newcommand{\bv}{\mathbf{v}}
\newcommand{\bx}{\mathbf{x}}

\newcommand{\bX}{\mathbf{X}}



\begin{document}
\graphicspath{{figures/}}

\title{Evolving-surface glacier flow using Stokes dynamics}

\author{Ed Bueler}

\maketitle

\thispagestyle{empty}
\bigskip

FIXME Main points:\footnote{version \today}
\renewcommand{\labelenumi}{\arabic{enumi}.}
\begin{enumerate}
\item mesh coordinates are part of the model state; we solve for them implicitly over a time step $\Delta t$
\item initially time-stepping is by backward Euler implemented by hand; later perhaps firedrake\_ts and BDF2 or such
\item at the start of the time step the domain in the $(s,t)$ plane occupied by ice is called $\Lambda$, and it has a mesh; these are the base domain and mesh
\item equations are for variables $\bu=\left<u,v\right>,p,r$; all of these are functions of $(s,t) \in \Lambda$
\item we will rewrite the weak-form integrals using change of coordinates
    $$(s,t) \mapsto (x,z), \qquad x(s,t)=s, z(s,t)=t+r(s,t)$$
\item ``new'' domain occupied by the ice will be the set
    $$\Omega = \{(x(s,t),z(s,t)) \,\big|\, (s,t) \in \Lambda\}$$
\item note that the $x$ coordinates of the mesh do not change; only vertical mesh motion
\item the $z$ value at the ice surface solves the time-discretized surface kinematical equation re-written into $(x,\zeta$ coordinates; this is a Robin boundary condition for $r$
\item in the interior of $\Lambda$, the difference $r(s,t)=z(s,t)-t$ solves the Laplace equation subject to boundary values as follows:
    \begin{itemize}
    \item Robin on surface from surface kinematical equation
    \item zero Dirichlet on base and inflow; these mesh nodes do not move
    \item Neuman on outflow
    \end{itemize}
\item numerical solver should check element orientation under change of coordinates (above); if the solver flips an element then it is bad; also check new element aspect ratio and (presumably) remesh if that is bad
\item the initial iterate for the (SNES-based) solver is clear: $\bu,p$ come from solution of previous time step, and $r$ starts at zero
\end{enumerate}


\section{Evolving surface Stokes model for glacier ice}

First we state the weak form of the Stokes model for ice flow, which is the conservation of momentum model, with glacier-suitable boundary conditions.    Next we describe the implicit moving-mesh scheme to solve the surface kinematical equation, which requires us to change variables in the integral defining the Stokes weak form.

The Glen-Stokes model for ice is described in \cite{GreveBlatter2009,JouvetRappaz2011}.  For these notes it applies on a 2D domain $\Omega$ occupied by the ice, in variables $(x,z)$ with $z$ at angle $\alpha$ to the vertical (below).  The domain $\Omega = \Omega_t \subset \RR^2$ which must have a piecewise smooth boundary so that we may apply the boundary conditions, and in fact it will evolve, but it is otherwise general.

Allowing any Glen exponent $n\ge 1$, the strong form model equations are:
\begin{align}
- \nabla \cdot \tau + \nabla p &= \rho \bg &&\text{\emph{stress balance}} \label{forcebalance} \\
\nabla \cdot \bu &= 0 &&\text{\emph{incompressibility}} \label{incompressible} \\
\tau &= B_n |D\bu|^{\frac{1}{n} - 1} D\bu  &&\text{\emph{viscosity-form flow law}} \label{viscflowlaw}
\end{align}
Here we have velocity $\bu$, pressure $p$, ice density $\rho$ (constant), acceleration of gravity $\bg = \left<g\sin\alpha,-g\cos\alpha\right>$ where $g>0$, deviatoric stress tensor $\tau$, and strain rate tensor $D\bu$.  Tensors $D\bu$ and $\tau$ are symmetric and have trace zero.  Recall $D\bu$ is the symmetric part of $\grad \bu$, $(D\bu)_{ij} = \frac{1}{2} \left(\grad\bu + \grad\bu^\top\right)$.  The full (Cauchy) stress tensor $\sigma$ is the deviatoric stress tensor $\tau$ minus the pressure, $\sigma = \tau - p\,I$, so equation \eqref{forcebalance} simply says $-\Div \sigma = \rho \bg$.  Note that $\tau = 2\nu D\bu$ where $\nu$ is the \emph{effective viscosity}, i.e.~$\nu=(1/2)B_n |D\bu|^{\frac{1}{n} - 1}$, $B_n = (A_n)^{-1/n}$ is the $n$-dependent \emph{ice hardness} in units $\text{Pa}\,\text{s}^{1/n}$, and the tensor norm is $|D\bu|^2 = \frac{1}{2} \trace\left((D\bu)^2\right) = \frac{1}{2} (D\bu)_{ij} (D\bu)_{ij}$.

From \eqref{viscflowlaw} we can eliminate $\tau$ from equation \eqref{forcebalance}, thereby rewriting the system in terms of velocity and pressure only:
\begin{align}
- \nabla \cdot \left(B_n |D\bu|^{\frac{1}{n} - 1} D\bu\right) + \nabla p &= \rho \mathbf{g} \label{stokes} \\
\Div \bu &= 0 \label{incompagain}
\end{align}
This system is the Glen-Stokes model, and a solution is a velocity-pressure pair $(\bu,p)$.

Certain glacier-suitable velocity and stress boundary conditions will be used here.  We assume that the base, top, inflow, and outflow boundaries can all be identified.  On the base we require no slip:
\begin{align}
\bu &= 0  &&\text{\emph{base}} \label{basebc} \\
\intertext{On the top we set a condition of zero applied stress, i.e.~$\sigma\hbn=0$:}
\left(B_n |D\bu|^{\frac{1}{n} - 1} D\bu - pI\right) \hbn &= 0  &&\text{\emph{top}} \label{topbc} \\
\intertext{The left-side inflow boundary has outward normal $\hbn=\left<-1,0\right>^\top$ in our case, and on this surface we set a nonzero inflow velocity:}
\bu &= \left<f(z),0\right>^\top  &&\text{\emph{inflow}} \label{inflowbc} \\
\intertext{The function $f(z)$ satisfies the slab-on-slope equations for a specific thickness $H_{\text{in}}$ at the inflow.  On the outflow boundary, where $\hbn=\left<1,0\right>^\top$, we set a nonzero hydrostatic normal stress using the (varying) ice thickness $H_{\text{out}}$ at the outflow:}
\left(B_n |D\bu|^{\frac{1}{n} - 1} D\bu - pI\right) \hbn &= C_{\text{out}} \big<- \rho g \cos\alpha (H_{\text{out}}-z), &&\text{\emph{outflow}} \label{outflowbc} \\
    &\qquad\qquad \rho g\sin\alpha (H_{\text{out}}-z)\big>^\top  \notag
\end{align}
The constant $C_{\text{out}} $ is adjusted so that the total applied stress is equal to its value for the input ice thickness $H_{\text{in}}$, thus $C_{\text{out}} = (H_{\text{in}}/H_{\text{out}})^2$.

The above equations determine the velocity and pressure in the ice from the boundary conditions, but they do not determine how the glacier changes shape.  For this we suppose that the domain on which the equations apply is the time-dependent set
\begin{equation}
\Omega^t = \left\{(x,z)\,\big|\, b(x) < z < h(x,t)\right\}  \label{Omegat}
\end{equation}
where $z=b(x)$ is the base elevation and $z=h(x,t)$ is the ice surface elevation.  Note the key fact that
\begin{equation}
h(x,t) \ge b(x).  \label{admissibility}
\end{equation}

The time-dependent surface $z=h(x,t)$ evolves according to the surface kinematical equation.  Using ice velocity $\bu=\left<u(x,z,t),w(x,z,t)\right>$ it is
\begin{equation}
h_t = a(x,t) - u(x,h,t) h_x + w(x,h,t) \label{surfacekinematical}
\end{equation}
Here $a(x,t)$ is the climatic mass balance in units of ice-equivalent $\text{m}\,\text{s}^{-1}$.  Informally, \eqref{surfacekinematical} determines the change in surface elevation $\Delta h \approx h_t\,\Delta t$ from the climatically added/removed ice $a\,\Delta t$ plus the component of the ice motion in the outward (upward) normal direction $\bn = \left<-h_x,1\right>$, so one may say $\Delta h \approx \left(a + \bn\cdot \bu|_{z=h}\right) \Delta t$.


\section{Weak form of the Stokes model} \label{sec:weakform}

The solution to the weak form is a pair $(\bu,p)$ with $\bu\in V_D$ and $p \in Q$, and function (Sobolev) spaces $V_D,Q$ are precisely-identified in \cite{JouvetRappaz2011}.  Test functions $\bv\in V_0$ and $q\in Q$ come from nearly the same spaces: $\bu\in V_D$ satisfies base and inflow boundary conditions \eqref{basebc} and \eqref{inflowbc}, respectively, while $\bv\in V_0$ is zero on those boundaries.

To give an initial definition of $F$ we multiply \eqref{stokes} by $\bv\in V_0$ and \eqref{incompagain} by $q\in Q$, then add and integrate-by-parts::
\begin{equation}
F(\bu,p;\bv,q) = -\int_{\partial\Omega} (\sigma \hbn)\cdot \bv + \int_\Omega \trace(\tau D\bv) - p (\nabla \cdot \bv) - \left(\nabla \cdot \bu\right) q - \rho \mathbf{g} \cdot \bv \label{nonfunctwo}
\end{equation}
(We have denoted $\sigma=\tau-pI$ for brevity.)  Now $\bu,\bv$ appear with at most first derivatives and $p,q$ appear without derivatives.  Next recall that $\bv\in V_0$ satisfies $\bv=0$ along the base and inflow surfaces.  Thus these parts of the integral over $\partial\Omega$ in \eqref{nonfunctwo} are zero.  Conditions \eqref{topbc}, \eqref{outflowbc} now completely eliminate the unknown solution $\bu,p$ from the boundary integral.  This yields our formula for the nonlinear functional:
\begin{align}
F(\bu,p;\bv,q) &= \int_\Omega B_n |D\bu|^{\frac{1}{n} - 1} D\bu\,:\,D\bv - p (\nabla \cdot \bv) - \left(\nabla \cdot \bu\right) q \label{defineF} \\
    &\qquad  - \int_\Omega \rho \mathbf{g} \cdot \bv - \int_{\{\text{outflow}\}} C_{\text{out}} \rho g \cos\alpha (h-z) v  \notag
\end{align}
The last two integrals can be regarded as source terms; if the inflow velocity is zero and these source terms are zero then the unique solution is $\bu=0$ and $p=0$.

The weak form of the Glen-Stokes model is the statement that, at the solution $\bu\in V_D$ and $p\in Q$,
\begin{equation}
F(\bu,p;\bv,q) = 0 \qquad \text{ for all } \bv\in V_0 \text{ and } q\in Q  \label{weak}
\end{equation}
This weak formulation is proven in \cite[Theorem 3.8]{JouvetRappaz2011} to be well-posed under reasonable assumptions about the domain $\Omega$ and boundary data which are satisfied in the cases we consider.


\section{The weak form of a backward Euler step}

FIXME


\small

\bigskip
\bibliography{simp}
\bibliographystyle{siam}

\end{document}
