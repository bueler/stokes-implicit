\documentclass[letterpaper,final,12pt,reqno]{amsart}

\usepackage[total={6.3in,9.2in},top=1.1in,left=1.1in]{geometry}

\usepackage{bm}
\usepackage{empheq}
\usepackage[dvipsnames]{xcolor}
\usepackage{graphicx}
\usepackage{verbatim,fancyvrb}
\usepackage{tikz}

% hyperref should be the last package we load
\usepackage[pdftex,
colorlinks=true,
plainpages=false, % only if colorlinks=true
linkcolor=blue,   % only if colorlinks=true
citecolor=Red,   % only if colorlinks=true
urlcolor=black     % only if colorlinks=true
]{hyperref}

\renewcommand{\baselinestretch}{1.05}

\newcommand{\ddt}[1]{\ensuremath{\frac{\partial #1}{\partial t}}}
\newcommand{\ddx}[1]{\ensuremath{\frac{\partial #1}{\partial x}}}
\newcommand{\ddy}[1]{\ensuremath{\frac{\partial #1}{\partial y}}}
\newcommand{\pp}[2]{\ensuremath{\frac{\partial #1}{\partial #2}}}
\renewcommand{\t}[1]{\texttt{#1}}
\newcommand{\Matlab}{\textsc{Matlab}\xspace}
\newcommand{\eps}{\epsilon}
\newcommand{\RR}{\mathbb{R}}

\newcommand{\grad}{\nabla}
\newcommand{\Div}{\nabla\cdot}
\newcommand{\trace}{\operatorname{tr}}

\newcommand{\hbn}{\hat{\mathbf{n}}}

\newcommand{\bg}{\mathbf{g}}
\newcommand{\bn}{\mathbf{n}}
\newcommand{\bu}{\mathbf{u}}
\newcommand{\bv}{\mathbf{v}}
\newcommand{\bx}{\mathbf{x}}

\newcommand{\bX}{\mathbf{X}}

\newcommand{\bzero}{\bm{0}}

\newtheorem{lemma}{Lemma}



\begin{document}
\title{Evolving-surface glacier flow using Stokes dynamics}

\author{Ed Bueler}

\maketitle

\thispagestyle{empty}
\bigskip

\section{Equations of ice flow in glaciers} \label{sec:strongform}

Our conservation of momentum model for ice flow is the set of shear-thinning (Glen law) Stokes equations with glacier-suitable boundary conditions.\footnote{version: \today}  This model is described in \cite{GreveBlatter2009,JouvetRappaz2011}.  For these notes it applies on a 2D domain $\Omega$ occupied by the ice, in variables $(x,z)$ with $z$ vertical.  The domain $\Omega \subset \RR^2$ which must have a piecewise smooth boundary so that we may apply the boundary conditions, and in fact it will evolve, but it is otherwise general.

Allowing any Glen exponent $n\ge 1$, the strong form model equations are:
\begin{align}
- \nabla \cdot \tau + \nabla p &= \rho \bg &&\text{\emph{stress balance}} \label{forcebalance} \\
\nabla \cdot \bu &= 0 &&\text{\emph{incompressibility}} \label{incompressible} \\
\tau &= B_n |D\bu|^{(1/n) - 1} D\bu  &&\text{\emph{viscosity-form flow law}} \label{viscflowlaw}
\end{align}
Here we have velocity $\bu$, pressure $p$, ice density $\rho$ (constant), deviatoric stress tensor $\tau$, and strain rate tensor $D\bu$.  The acceleration of gravity is $\bg = \left<g\sin\alpha,-g\cos\alpha\right>$ where $g>0$ and $\alpha$ is an optional angle allowed for (e.g.) exact solutions.

Regarding tensors, recall $D\bu$ is the symmetric part of $\grad \bu$, i.e.~$(D\bu)_{ij} = \frac{1}{2} \left(\grad\bu + \grad\bu^\top\right)$, and the tensor norm is $|D\bu|^2 = \frac{1}{2} \trace\left((D\bu)^2\right) = \frac{1}{2} (D\bu)_{ij} (D\bu)_{ij}$.  Since $D\bu$ is symmetric and has trace zero by \eqref{incompressible}, the same holds for $\tau$.  The full (Cauchy) stress tensor $\sigma$ is the deviatoric part $\tau$ minus the pressure, $\sigma = \tau - p\,I$, so equation \eqref{forcebalance} states $-\Div \sigma = \rho \bg$.  For linear Stokes equations we would replace \eqref{viscflowlaw} by $\tau = 2\nu D\bu$, thus we say here that $\nu=(1/2)B_n |D\bu|^{(1/n) - 1}$ is the \emph{effective viscosity}.  Note $B_n = (A_n)^{-1/n}$ is the $n$-dependent \emph{ice hardness} in units $\text{Pa}\,\text{s}^{1/n}$.

From \eqref{viscflowlaw} we can eliminate $\tau$ from equation \eqref{forcebalance}, thereby rewriting the system in terms of velocity and pressure only:
\begin{align}
- \nabla \cdot \left(B_n |D\bu|^{(1/n) - 1} D\bu\right) + \nabla p &= \rho \mathbf{g} \label{stokes} \\
\Div \bu &= 0 \label{incompagain}
\end{align}
This system is the Glen-Stokes model, and a solution is a velocity-pressure pair $(\bu,p)$.

Certain glacier-suitable velocity and stress boundary conditions will be used here.  We assume that the base, top, inflow, and outflow boundaries can all be identified.  Note that the inflow and outflow boundaries may be empty but we assume that the base and top are nonempty.  On the base, for this first version of the model, we require no slip:
\begin{align}
\bu &= \bzero  &&\text{\emph{base}} \label{basebc} \\
\intertext{On the top we set a condition of zero applied stress, i.e.~$\sigma\hbn=0$ where $\hbn$ is the outward normal:}
\left(B_n |D\bu|^{\frac{1}{n} - 1} D\bu - pI\right) \hbn &= \bzero  &&\text{\emph{top}} \label{topbc} \\
\intertext{One the inflow boundary we require a prescribed inflow velocity:}
\bu &= \bu_{\text{in}}  &&\text{\emph{inflow}} \label{inflowbc} \\
\intertext{On the outflow boundary we set a nonzero hydrostatic normal stress determined by the ice thickness $H_{\text{out}}$ at the outflow, which must be known:}
\left(B_n |D\bu|^{\frac{1}{n} - 1} D\bu - pI\right) \hbn &= -\rho \bg (H_{\text{out}}-z) &&\text{\emph{outflow}} \label{outflowbc}
\end{align}
If $\alpha=0$ and the boundary is vertical then this boundary condition corresponds to zero traction tangent to the boundary and hydrostatic pressure applied normally.

The above equations determine the velocity and pressure in the ice from the boundary conditions, and from the geometry of the ice domain $\Omega$, but they do not determine how the glacier changes shape.  One aspect of conservation of mass is already stated above, namely the incompressibility relation in the interior of $\Omega$, but to evolve the glacier geometry we will need to add the conservation of mass statement on $\partial\Omega$, namely the surface kinematical equation.  To present this equation we suppose that the domain is time-dependent:
\begin{equation}
\Omega^t = \left\{(x,z)\,\big|\, b(x) < z < h(x,t)\right\}.  \label{Omegat}
\end{equation}
Here $z=b(x)$ is the base elevation and $z=h(x,t)$ is the ice surface elevation, and note the key fact that
\begin{equation}
h(x,t) \ge b(x).  \label{admissibility}
\end{equation}
Denoting the ice velocity components as $\bu=\left<u(x,z,t),w(x,z,t)\right>$, the surface kinematical equation is
\begin{equation}
h_t = a(x,h,t) - u(x,h,t) h_x + w(x,h,t) \label{surfacekinematical}
\end{equation}
Here $a(x,z,t)$ is the (modeled) climatic mass balance in (ice-equivalent) units $\text{m}\,\text{s}^{-1}$.  Informally, \eqref{surfacekinematical} changes the surface elevation $\Delta h \approx h_t\,\Delta t$ using the climatically added/removed ice $a\,\Delta t$ plus the component of the ice motion in the outward (upward) normal direction $\hbn = \left<-h_x,1\right>$.  One may say that the time-dependent surface $z=h(x,t)$ evolves by steps $\Delta h \approx \left(a + \hbn\cdot \bu|_{z=h}\right) \Delta t$, but in fact we will solve \eqref{surfacekinematical} implicitly, that is, coupled with the Stokes equations \eqref{stokes}, \eqref{incompagain} and the other boundary conditions.


\section{Weak form of the decoupled Glen-Stokes equations} \label{sec:weakformstokes}

At any time $t$, by well-posedness \cite{JouvetRappaz2011}, if the ice geometry $\Omega^t$ is known then the solution to Glen-Stokes model \eqref{stokes}, \eqref{incompagain} is a unique pair of functions $(\bu,p)$.  That is, in a Stokes model the slowness of the fluid ($\text{Re}=0$) implies that the boundary stresses and body force determine its velocity and pressure fields without any ``memory'' of prior states; there is no influence from inertia.  For this situation, as is standard in the literature, one may write a weak form for the variable-viscosity momentum-balance equations, decoupled from the mass conservation boundary equation \eqref{surfacekinematical}.  We state this decoupled weak form so as to set notation and have a familiar starting point, before describing the coupled, implicit geometry evolution in the next section.

For the weak form we seek a solution $\bu\in V_D \subset W^{1,p}(\Omega^t)$ and $p \in Q=L^{p'}(\Omega^t)$.  Test functions $\bv\in V_0 \subset W^{1,p}(\Omega^t)$ and $q\in Q$ come from nearly the same spaces: $\bu\in V_D$ satisfies base and inflow boundary conditions \eqref{basebc} and \eqref{inflowbc}, respectively, while $\bv\in V_0$ is zero on those boundaries.  FIXME: clarify function spaces

To give an initial definition of the nonlinear functional which defines the weak form, we multiply \eqref{stokes} by $\bv\in V_0$ and \eqref{incompagain} by $q\in Q$, then add and integrate-by-parts:
\begin{equation}
\tilde F = -\int_{\partial\Omega^t} (\sigma \hbn)\cdot \bv\,dS + \int_{\Omega^t} \tau \,:\,D\bv - p (\nabla \cdot \bv) - \left(\nabla \cdot \bu\right) q - \rho \mathbf{g} \cdot \bv \,dx dz \label{nonfunctwo}
\end{equation}
where $dS$ is the arclength element along $\partial\Omega^t$, and denoting $\sigma=\tau-pI$ for brevity.  Note that $\bu,\bv$ appear with at most first derivatives and $p,q$ appear without derivatives.

By definition, $\bv\in V_0$ satisfies $\bv=\bzero$ along the base and inflow surfaces.  Thus these parts of the integral over $\partial\Omega$ in \eqref{nonfunctwo} are zero.  The stress-free condition \eqref{topbc} on the top surface and the outflow condition \eqref{outflowbc} on the outflow boundary denoted $\partial_O \Omega^t$, now completely eliminate the unknowns from the boundary integral, yielding a final formula for $\tilde F$:
\begin{align}
\tilde F &= \int_{\Omega^t} B_n |D\bu|^{\frac{1}{n} - 1} D\bu\,:\,D\bv - p (\nabla \cdot \bv) - \left(\nabla \cdot \bu\right) q \,dx dz \label{definetildeF} \\
    &\qquad  - \int_{\Omega^t} \rho \mathbf{g} \cdot \bv \,dx dz + \int_{\partial_O \Omega^t} \rho (H_{\text{out}}-z) \bg \cdot \bv \,dS \notag
\end{align}
The last two integrals can be regarded as source terms.  In fact, if the inflow velocity is zero and these source terms are also zero, for instance because gravity is turned off ($\bg=\bzero$), then the unique solution is $\bu=\bzero$ and $p=0$.  The weak form is the statement that $\bu\in V_D$ and $p\in Q$ satisfy
\begin{equation}
\tilde F(\bu,p;\bv,q) = 0 \qquad \text{ for all } \bv\in V_0 \text{ and } q\in Q  \label{weak}
\end{equation}

An equivalent weak formulation is proven in \cite[Theorem 3.8]{JouvetRappaz2011} to be well-posed under reasonable assumptions about the domain $\Omega$ and boundary data which are satisfied in the cases we consider.


\section{Implicit time-discretization and the icy domain update} \label{sec:implicitstep}

We propose an implicit domain-updating scheme to simultaneously solve the Glen-Stokes equations and the surface kinematical equation.  In this scheme we compute a greatly-simplified surrogate of the ice \emph{strain}, i.e.~the deformation or displacement, during the time step, implicitly from the simultaneous principles of mass and momentum conservation.  Specifically, we solve for a vertical-only displacement function which is defined within the updated icy domain.  At each time step the current domain is, essentially, the reference configuration from which we compute the updated configuration.  In writing-out the equations of this scheme, stated using a time-discretized but continuous-space weak formulation, a significant task is to change variables in the integrals which define the Stokes weak form.  In the finite element solution of the resulting equations, considered in the next section, the domain update becomes a vertical-only mesh displacement computed simultaneously with the discrete velocity and pressure fields.

Initially we will describe the time-stepping using a backward Euler scheme, a somewhat inflexible presentation.  Better implicit schemes exist, and in later work we could state the problem as a differential algebraic equation (DAE) and apply a higher-order scheme like a second-order backward-differentiation formula (BDF2) \cite{AscherPetzold1998}, and use tools like PETSc's TS object \cite{Balayetal2020,BuelerBook} or \texttt{firedrake-ts}.\footnote{\url{https://github.com/IvanYashchuk/firedrake-ts}}

Let $t_{n-1}$ and $t_n$ be consecutive times with step $\Delta t = t_n - t_{n-1} > 0$.  Suppose the model's state, namely the (current) ice geometry, is known at time $t_{n-1}$.  Denote this ice-filled domain as $\Lambda = \Omega^{t_{n-1}} \subset \RR^2$.  We will approximate the updated (new) domain $\Omega^n = \Omega^{t_n} \subset \RR^2$ using the surface kinematical equation \eqref{surfacekinematical}, the surface mass balance $a(x,z,t)$, and the Glen-Stokes equations \eqref{weak}.

The coordinates on the current domain $\Lambda$ are denoted $(r,s)$.  We update the region via a change of coordinates $(r,s) \mapsto (x,z)$,
\begin{equation}
\Lambda \stackrel{\Delta t}{\to} \Omega^n: \qquad x(r,s)=r, \quad z(r,s)=s+c(r,s), \label{changecoords}
\end{equation}
so the new domain $\Omega^n$ is an image of $\Lambda$:
\begin{equation}
\Omega^n = \{(x,z)=(r,s+c(r,s)) \,\big|\, (r,s) \in \Lambda\}.  \label{updateddomain}
\end{equation}
(Note that the horizontal coordinate is not changed ($x=r$) but the $z$-coordinate is nontrivial.)  Here $c(r,s)$ is a scalar function which is to be determined (below).  In fact we will solve coupled PDEs on $\Lambda$, including a problem for the scalar vertical displacement function $c(r,s)$, and the coupled solution determines $\Omega^n$ plus the ice velocity and pressure.  This is sketched in Figure \ref{fig:domainupdate}.

\begin{figure}[h]

FIXME

\caption{The time $t_{n-1}$ (``current'') domain $\Lambda$ is the reference configuration for the time $t_n$ (``new'') domain $\Omega^n$.}
\label{fig:domainupdate}
\end{figure}

The top of the icy region will be described by a surface elevation function.  This requires we acknowledge a ``shallowness'' assumption, namely that for every $x$-coordinate value there are well-defined surface and base elevation values.  That is, from now on we assume that
\begin{equation}
I(r) = \{s\,:\,(r,s) \in \Lambda\} \subset \RR \, \text{ is a single interval}\label{intervalassume}
\end{equation}
for every $r$.  In fact $I(r)$ is open, because $\Lambda$ is itself open, but this is of no importance in the following.  While statement \eqref{intervalassume} is an assumption about $\Lambda$, the current icy domain, once we have solved all the upcoming equations the same must also hold for $\Omega^n$, that is, so that time-stepping may proceed.  Now let
    $$\eta(r) = \sup I(r), \qquad b(r) = \inf I(r)$$
define the ice-surface and base elevation functions for the current domain $\Lambda$.  Given \eqref{updateddomain}, the surface elevation for the new region is
\begin{equation}
h(r) = \eta(r) + c(r,\eta(r)),  \label{newsurfaceelevation}
\end{equation}
but note this is defined for $r$ in $\Lambda$ (i.e.~not $x$ in $\Omega^n$).  The value $c(r,\eta(r))$ is defined in a trace sense \cite{Evans2010}; see below for the Sobolev space which contains $c$.

The change of coordinates will be differentiable within $\Lambda$.  (This is because the nontrivial part, the function $c$, will solve an elliptic PDE; see below.)  From \eqref{changecoords} the Jacobian of the change is
\begin{equation}
J = \begin{pmatrix} \partial x / \partial r & \partial x / \partial s \\ {\large\strut} \partial z / \partial r & \partial z / \partial s \end{pmatrix} = \begin{pmatrix} 1 & 0 \\ \partial c/\partial r & 1+(\partial c/\partial s) \end{pmatrix}. \label{jacchange}
\end{equation}
For a generic, smooth, scalar-valued function $\tilde f(x,z)$ defined on $\Omega^n$ we define a new function on $\Lambda$,
    $$f(r,s) = \tilde f(r,s+c(r,s)),$$
and partial derivatives transform as
\begin{equation}
\begin{bmatrix} \partial \tilde f / \partial x \\ {\large\strut} \partial \tilde f / \partial z\end{bmatrix} = (J^\top)^{-1} \begin{bmatrix} \partial f / \partial r \\ {\large\strut} \partial f / \partial s\end{bmatrix} = \begin{pmatrix} 1 & \ell \\ 0 & k \end{pmatrix} \begin{bmatrix} \partial f / \partial r \\ {\large\strut} \partial f / \partial s\end{bmatrix} \label{changederivatives}
\end{equation}
where
\begin{equation}
j(r,s) = 1+\frac{\partial c}{\partial s}, \qquad k(r,s) = j(r,s)^{-1}, \qquad \ell(r,s) = - \frac{\partial c}{\partial r}(r,s) j(r,s)^{-1}. \label{definejkl}
\end{equation}

Note that $j(r,s)$ is the Jacobian determinant.  For stability we require that the change of coordinates not cause a local flip in orientation, though degeneration is allowed.  That is, we will require $j\ge 0$ or equivalently
\begin{equation}
\frac{\partial c}{\partial s} \ge -1. \label{differentialVI}
\end{equation}
(We reconsider the meaning of this restriction in \ref{sec:inequalities}.)

The time derivative $h_t$ in surface kinematical equation \eqref{surfacekinematical} will be approximated by $(h^n(r) - h(r))/\Delta t$.  Let
\begin{equation}
a^n(r) = a\left(r,\eta(r) + c(r,\eta(r)),t_n\right) \label{massbalance}
\end{equation}
be the mass balance computed at the updated surface location and the updated time.  Using \eqref{newsurfaceelevation}, our $O(\Delta t)$ approximation of \eqref{surfacekinematical}, along the top surface $s=\eta(r)$ of $\Lambda$ is
\begin{equation}
\frac{c(r,\eta(r))}{\Delta t} = a^n(r) - u(r,\eta(r))\,\eta'(r) + w(r,\eta(r)). \label{surfaceimplicit}
\end{equation}
Equation \eqref{surfaceimplicit} is a boundary condition at each top location $(r,\eta(r))$ on $\Lambda$.  In fact $\bu(r,s)=\left<u(r,s),w(r,s)\right>$ will also solve the Stokes equations (see below) on $\Lambda$, so condition \eqref{surfaceimplicit} couples the determination of the domain $\Omega^n$ with velocity and pressure.  Condition \eqref{surfaceimplicit} also couples surface elevation and mass balance during the solve, but a simplification is to use the time-splitting approximation $a^n \approx a(r,\eta(r),t_n)$ instead of \eqref{massbalance}.

We propose that the vertical displacement function $c(r,s)$, defined on the current domain $\Lambda$, solves a PDE problem, part of which is the boundary condition \eqref{surfaceimplicit}.  In strong form, $c$ will solve the Laplace equation with these boundary conditions:
\begin{align}
        \grad^2 c &= 0 &&\text{in } \Lambda \label{claplace} \\
                c &= \Delta t\,\left(a^n - u \eta' + w\right) &&\emph{top} \notag \\
                c &= 0 &&\text{\emph{base} \& \emph{inflow}} \notag \\
\grad c\cdot \hbn &= 0 &&\emph{outflow} \notag
\end{align}
Dirichlet conditions apply on most of the boundary, but a Neumann condition applies on the outflow (if present).  Assuming that $\Lambda$ is a well-behaved domain, this boundary-value problem is known to be well-posed in the \emph{decoupled} context where $a^n,u,w$ are given functions (with mild regularity assumptions) \cite{Evans2010}, but we intend to solve it in the coupled context where the Dirichlet condition on the top actually relates unknowns $c,u,w$.

The weak form of problem \eqref{claplace} can be stated as follows.  Noting that the outflow has a natural condition, we define the Laplacian bilinear form
\begin{equation}
a(c;e) = \int_\Lambda \frac{\partial c}{\partial r} \frac{\partial e}{\partial r} + \frac{\partial c}{\partial s} \frac{\partial e}{\partial s}\,dr ds  \label{asurface}
\end{equation}
and say that $c$ solves the (decoupled) weak form of \eqref{claplace} if $c \in W_D^{1,2}(\Lambda)$ and $e \in W_0^{1,2}(\Lambda)$ solves $a(c;e)=0$ for all $e \in W_0^{1,2}(\Lambda)$.  Note $c$ satisfies the nonhomogeneous Dirichlet conditions on the top [FIXME: HOW GIVEN \texttt{EquationBC}?], while $e$ is zero on the top, and both are zero on the base and inflow.

FIXME a fully-implicit version of \eqref{surfaceimplicit} would be
\begin{equation}
\frac{c(r,\eta(r))}{\Delta t} = a^n(r) - u(r,\eta(r))\,\frac{d}{dr}(h(r)) + w(r,\eta(r)). \label{surfaceimplicitfully}
\end{equation}
unfortunately this is more difficult because it involves an ``oblique derivative'' condition on \eqref{claplace}; but it can be tried out fairly easily with firedrakes \texttt{EquationBC}


\section{Weak form of the coupled equations} \label{sec:weakformcoupled}

The change of variables defined in the last section will rewrite the weak form integrals.  For example, consider a generic scalar-valued $L^1$ function $\tilde f(x,z)$ defined on $\Omega^n$.  By the change of variables theorem we have
\begin{equation}
\int_{\Omega^n} \tilde f(x,z)\,dx dz = \int_\Lambda f(r,s) \, j(r,s)\,dr ds, \label{changeintegral}
\end{equation}
with weight $j(r,s)$ defined in \eqref{definejkl} and $f(r,s) = \tilde f(r,s+c(r,s))$.  Also note the following expansions of velocity derivatives (strain rates) with respect to $(x,z)$ which follow from \eqref{changederivatives} and \eqref{definejkl}.  Also, we use component notation $\bu = \left<u_0,u_2\right>$, compatible with coordinate notation $(x,z)$ on $\Omega$:
\begin{align*}
\grad \cdot \tilde \bu &= \frac{\partial u_0}{\partial r} + \ell \frac{\partial u_0}{\partial s} + k \frac{\partial u_2}{\partial s} \\
D \tilde \bu &= \begin{pmatrix} \partial u_0/\partial r + \ell (\partial u_0/\partial s) & \gamma \\
  {\large\strut} \gamma & k (\partial u_2/\partial s)\end{pmatrix} \\
|D \tilde \bu|^2 &= \frac{1}{2}\left(\frac{\partial u_0}{\partial r} + \ell \frac{\partial u_0}{\partial s}\right)^2 + \gamma^2 + \frac{1}{2}\left(k \frac{\partial u_2}{\partial s}\right)^2
\end{align*}
where
    $$\gamma = \frac{1}{2} \left(k \frac{\partial u_0}{\partial s} + \frac{\partial u_2}{\partial r} + \ell \frac{\partial u_2}{\partial s}\right)$$
is a notational simplification only.

FIXME state the weak form using independent variables $r,s$ and integrals over $\Lambda$; equations are for solution functions $\bu,p,c$ with test functions $\bv,q,e$

From weak forms \eqref{definetildeF} and \eqref{asurface}, the proposed [DRAFT] weak form is
\begin{align}
F(\bu,p,c;\bv,q,e) &= \int_\Lambda B_n |D\bu|^{\frac{1}{n} - 1} D\bu\,:\,D\bv\, j\,dr ds [FIXME] \label{defineF} \\
    &\quad  - \int_\Lambda p \left(j \frac{\partial v_0}{\partial r} - \frac{\partial c}{\partial r} \frac{\partial v_0}{\partial s} + \frac{\partial v_2}{\partial s}\right) \,dr ds \notag \\
    &\quad - \int_\Lambda q \left(j \frac{\partial u_0}{\partial r} - \frac{\partial c}{\partial r} \frac{\partial u_0}{\partial s} + \frac{\partial u_2}{\partial s}\right)\,dr ds \notag \\
    &\quad  - \int_\Lambda \rho \mathbf{g} \cdot \bv \, j\,dr ds \notag \\
    &\quad + \int_{\partial_O \Lambda} \rho (H_{\text{out}}-s-c) \bg \cdot \bv \,dS \notag \\
    &\quad + \int_\Lambda \frac{\partial c}{\partial r} \frac{\partial e}{\partial r} + \frac{\partial c}{\partial s} \frac{\partial e}{\partial s}\,dr ds \notag
\end{align}
FIXME in above: $|D\bu|$ and $D\bu:D\bv$ need expansion; how to regularize that term; how to write weak form for $c$ when we are actually going to use \texttt{EquationBC}?

The solution of the weak form is a list of three functions, $\bu = \left<u_0,u_2\right> \in V_D$, $p\in L^{p'}(\Lambda)$, and $c\in W^{1,2}_D(\Lambda)$, such that
\begin{equation}
F(\bu,p,c;\bv,q,e) = 0 \text{ for all } \bv = \left<v_0,v_2\right> \in V_0, \, q\in L^{p'}(\Lambda), \text{ and } e \in W^{1,2}_0(\Lambda).
\end{equation}
All trial and test functions are functions of $r$ and $s$ in $\Lambda$.

FIXME result is following block structure


\section{Inequality constraints and mass accounting} \label{sec:inequalities}

\begin{lemma}
Assume that the ice thickness is well defined at time $t_{n-1}$, that is, assume
\eqref{intervalassume}.  If inequality \eqref{differentialVI} holds then the updated ice surface elevation
    $$h(r) = \sup_{s\in I(r)}\{z(r,s)\}$$
satisfies \eqref{admissibility}, i.e.~$h(r)\ge b(r)$.
\end{lemma}

\begin{proof}
FIXME? This is the fundamental theorem of calculus:
    $$h(r) - b(r) = \int_{I(r)} 1\,ds \le \int_{I(r)} - \frac{\partial c}{\partial s}\,ds$$
\end{proof}


\section{Finite element approximation}  \label{sec:finiteelement}

FIXME: numerical solver should check element orientation under change of coordinates (above); if the solver flips an element then it is bad; also check new element aspect ratio and (presumably) remesh if that is bad; the initial iterate for the (SNES-based) solver is clear: $\bu,p$ come from solution of previous time step, and $b$ starts at zero



\small

\bigskip
\bibliography{simp}
\bibliographystyle{siam}

\end{document}
