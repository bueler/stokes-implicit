\documentclass[letterpaper,final,12pt,reqno]{amsart}

\usepackage[total={6.3in,9.2in},top=1.1in,left=1.1in]{geometry}

\usepackage{bm}
\usepackage{empheq}
\usepackage[dvipsnames]{xcolor}
\usepackage{graphicx}
\usepackage{verbatim,fancyvrb}
\usepackage{tikz}

% hyperref should be the last package we load
\usepackage[pdftex,
colorlinks=true,
plainpages=false, % only if colorlinks=true
linkcolor=blue,   % only if colorlinks=true
citecolor=Red,   % only if colorlinks=true
urlcolor=black     % only if colorlinks=true
]{hyperref}

\renewcommand{\baselinestretch}{1.05}

\newcommand{\ddt}[1]{\ensuremath{\frac{\partial #1}{\partial t}}}
\newcommand{\ddx}[1]{\ensuremath{\frac{\partial #1}{\partial x}}}
\newcommand{\ddy}[1]{\ensuremath{\frac{\partial #1}{\partial y}}}
\newcommand{\pp}[2]{\ensuremath{\frac{\partial #1}{\partial #2}}}
\renewcommand{\t}[1]{\texttt{#1}}
\newcommand{\Matlab}{\textsc{Matlab}\xspace}
\newcommand{\eps}{\epsilon}
\newcommand{\RR}{\mathbb{R}}

\newcommand{\grad}{\nabla}
\newcommand{\Div}{\nabla\cdot}
\newcommand{\trace}{\operatorname{tr}}

\newcommand{\hbn}{\hat{\mathbf{n}}}

\newcommand{\bg}{\mathbf{g}}
\newcommand{\bn}{\mathbf{n}}
\newcommand{\bu}{\mathbf{u}}
\newcommand{\bv}{\mathbf{v}}
\newcommand{\bx}{\mathbf{x}}

\newcommand{\bX}{\mathbf{X}}

\newcommand{\bzero}{\bm{0}}

\newtheorem{lemma}{Lemma}



\begin{document}
\title{Evolving-surface glacier flow using Stokes dynamics}

\author{Ed Bueler}

\maketitle

\thispagestyle{empty}
\bigskip

\section{Equations of ice flow in glaciers} \label{sec:strongform}

The standard description of ice flow in glaciers is the shear-thinning (Glen law) Stokes model \cite{GreveBlatter2009,JouvetRappaz2011}.\footnote{version: \today}  We will apply this model on a time-dependent 2D domain, the region occupied by the ice, using time $t>0$ and spatial variables $(x,z)$, with $z$ vertical.  The evolving domain $\Omega^t \subset \RR^2$ must have a piecewise smooth boundary so that we may apply the boundary conditions, but it is otherwise general.

Allowing any Glen exponent $n\ge 1$, the strong form model equations are:
\begin{align}
- \nabla \cdot \tau + \nabla p &= \rho \bg &&\text{\emph{stress balance}} \label{forcebalance} \\
\nabla \cdot \bu &= 0 &&\text{\emph{incompressibility}} \label{incompressible} \\
\tau &= B_n |D\bu|^{(1/n) - 1} D\bu  &&\text{\emph{viscosity-form flow law}} \label{viscflowlaw}
\end{align}
The fields are velocity $\bu$, pressure $p$, deviatoric stress tensor $\tau$, and strain rate tensor $D\bu$.  We take the ice density $\rho$ and the acceleration of gravity $\bg = \left<g\sin\alpha,-g\cos\alpha\right>$, where $g>0$ and $\alpha$ is an optional angle allowed for (e.g.) exact solutions, to be constant.

Regarding tensors and notation, first recall that $D\bu$ is the symmetric part of $\grad \bu$, i.e.~$(D\bu)_{ij} = \frac{1}{2} \left(\grad\bu + \grad\bu^\top\right)$.  The tensor norm in \eqref{viscflowlaw} satisfies $|D\bu|^2 = \frac{1}{2} \trace\left((D\bu)^2\right) = \frac{1}{2} (D\bu)_{ij} (D\bu)_{ij}$.  The full (Cauchy) stress tensor $\sigma$ is the deviatoric part $\tau$ minus the pressure, i.e.~$\sigma = \tau - p\,I$, and so equation \eqref{forcebalance} says $-\Div \sigma = \rho \bg$.  For the linear Stokes equations, i.e.~the $n=1$ case, one would traditionally write \eqref{viscflowlaw} as $\tau = 2\nu D\bu$ \cite{Elmanetal2014}, thus one says in glaciology that $\nu=(1/2)B_n |D\bu|^{(1/n) - 1}$ is the \emph{effective viscosity}, a quantity which generally varies nonlinearly with the solution of the equations.  Also $B_n$ is the $n$-dependent \emph{ice hardness} in units $\text{Pa}\,\text{s}^{1/n}$; it is sometimes written $B_n = (A_n)^{-1/n}$ in terms of the \emph{ice softness} $A_n$.  Finally, because $D\bu$ is symmetric, and because it has trace zero by \eqref{incompressible}, i.e.~$\trace(D\bu)=\nabla \cdot \bu = 0$, by equation \eqref{viscflowlaw} the same two properties hold for $\tau$.

Using \eqref{viscflowlaw} we can eliminate $\tau$ from equation \eqref{forcebalance}, thereby rewriting the system in terms of velocity and pressure only:
\begin{align}
- \nabla \cdot \left(B_n |D\bu|^{(1/n) - 1} D\bu\right) + \nabla p &= \rho \mathbf{g} \label{stokes} \\
\Div \bu &= 0 \label{incompagain}
\end{align}
This system is the Glen-Stokes model.  A solution is a velocity-pressure pair $(\bu,p)$, from which one may derive the strain rates $D\bu$ and thus the stresses $\tau$ and/or $\sigma = \tau - p\,I$.

Certain glacier-suitable velocity and stress boundary conditions will be used here.  We assume that the base, top, inflow, and outflow boundaries of the glacier, i.e.~of $\Omega^t$ at each time, can all be identified, but in fact only the base and top surfaces (curves in the 2D case) are assumed to have positive measure.   On the base we require no slip:\footnote{It is expected that we will allow sliding laws for the ice base  \cite{JouvetRappaz2011} at some later time.}
\begin{align}
\bu &= \bzero  &&\text{\emph{base}} \label{basebc} \\
\intertext{On the top we set a condition of zero applied stress, i.e.~$\sigma\hbn=0$ where $\hbn$ is the outward normal:}
\left(B_n |D\bu|^{\frac{1}{n} - 1} D\bu - pI\right) \hbn &= \bzero  &&\text{\emph{top}} \label{topbc} \\
\intertext{On the inflow boundary we prescribe a velocity which should satisfy $\bu_{\text{in}}\cdot \hbn < 0$, where $\hbn$ is the outward unit normal of $\Omega^t$:}
\bu &= \bu_{\text{in}}  &&\text{\emph{inflow}} \label{inflowbc} \\
\intertext{On the outflow boundary we set a nonzero hydrostatic normal stress determined by the thickness $H_{\text{out}}$ of the ice outside of $\Omega^t$:}
\left(B_n |D\bu|^{\frac{1}{n} - 1} D\bu - pI\right) \hbn &= -\rho \bg (H_{\text{out}}-z) &&\text{\emph{outflow}} \label{outflowbc}
\end{align}
If there is no ice outside $\Omega^t$ then $H_{\text{out}}=0$.  If $\alpha=0$ and the boundary is vertical then the outflow boundary condition corresponds to zero traction tangent to the boundary and hydrostatic pressure applied normally.

By the well-posedness of the above model \cite{JouvetRappaz2011}, if at any time $t$ the ice geometry $\Omega^t$ is known then the solution to the Glen-Stokes model \eqref{stokes}--\eqref{outflowbc} is a unique pair of functions $(\bu,p)$.  That is, in a Stokes model the slowness of the fluid ($\text{Re}=0$) implies that the boundary stresses and body force determine its velocity and pressure fields without any ``memory'' of prior states; there is no influence from inertia.

The above equations determine the velocity and pressure within the ice from the boundary conditions, and from the geometry of the ice domain $\Omega^t$, but they do not describe the process by which the glacier changes shape.  This process, namely the \emph{surface kinematical equation}, states conservation of mass on the boundary $\partial\Omega^t$ of the icy domain.  (One form of conservation of mass is already stated above, namely the incompressibility relation in the interior of $\Omega^t$.)  To present this equation we suppose the following time-dependent form for the domain:
\begin{equation}
\Omega^t = \left\{(x,z)\,\big|\, b(x) < z < h(x,t)\right\}.  \label{Omegat}
\end{equation}
Here $z=b(x)$ is the base elevation, assumed time-independent for simplicity, and $z=h(x,t)$ is the ice surface elevation, and note the key fact that
\begin{equation}
h(x,t) \ge b(x).  \label{admissibility}
\end{equation}

Thus the top of the icy region is described by a surface elevation function $h(x,t)$.  We acknowledge that this represents a kind of ``shallowness'' assumption, namely that for every $x$-coordinate value there are well-defined surface and base elevation values.  That is, from now on we are assuming, in agreement with almost all ice sheet and glacier modeling literature, that
\begin{equation}
I(x,t) = \{z\,:\,(x,z) \in \Omega^t\} \subset \RR \, \text{ is always a single interval}.\label{intervalassume}
\end{equation}
In fact $I(x,t)$ is open, because $\Omega^t$ is itself open (but this is of no importance in the following).  Note that expressions
    $$h(x,t) = \sup I(x,t), \qquad b(x) = \inf I(x,t)$$
define the ice-surface and base elevation functions.

Denoting the velocity components as $\bu=\left<u(x,z,t),w(x,z,t)\right>$, the surface kinematical equation is
\begin{equation}
h_t = a(x,h,t) - u(x,h,t) h_x + w(x,h,t) \label{surfacekinematical}
\end{equation}
Here $a(x,z,t)$ is the (modeled) climatic mass balance in (ice-equivalent) units $\text{m}\,\text{s}^{-1}$.\footnote{Modeling melting or freeze-on at the ice base would add a basal kinematic equation \cite{Aschwandenetal2012}.}

Informally, \eqref{surfacekinematical} describes how the top surface elevation is updated, $\Delta h \approx h_t\,\Delta t$, by using the climatically added or removed ice $a\,\Delta t$, plus the component of the ice motion in the outward (upward) normal direction $\hbn = \left<-h_x,1\right>$.  In fact, models often evolve the time-dependent surface $z=h(x,t)$ by explicit steps $\Delta h \approx \left(a + \hbn\cdot \bu|_{z=h}\right) \Delta t$.  However, we will solve \eqref{surfacekinematical} implicitly, that is, coupled with the Stokes equations \eqref{stokes}, \eqref{incompagain}, and with the other boundary conditions.

As is standard in the literature, one may write a weak form for the above variable-viscosity momentum-balance equations \eqref{stokes}--\eqref{outflowbc}, but decoupled from the mass conservation boundary equation \eqref{surfacekinematical}.  We state this decoupled weak form in order to set notation and have a familiar starting point, before describing the coupled, implicit geometry evolution in the next section.

Let $p=1+1/n$, $p'=(1-p^{-1})^{-1}=p/(p-1)$, and fix $t>0$.  We will seek a weak-form solution from the following spaces:
\begin{align*}
\bu \in V_D &= \left\{\bv \in W^{1,p}(\Omega^t)\,:\,\bv\big|_{\text{base}}=\bzero \text{ and } \bv\big|_{\text{inflow}} = \bu_{\texttt{in}}\right\}, \quad p \in Q =L^{p'}(\Omega^t).
\end{align*}
Test functions $\bv$ and $q$ come from nearly the same spaces, with $\bv\big|_{\text{base}} =0$ and $\big|_{\text{inflow}} = 0$, namely $\bv \in V_0 \subset W^{1,p}(\Omega^t)$, and $q \in Q$ itself.

To derive the nonlinear functional which defines the weak form, we first multiply \eqref{stokes} by $\bv\in V_0$ and \eqref{incompagain} by $q\in Q$, then add and integrate-by-parts:
\begin{equation}
\tilde F = -\int_{\partial\Omega^t} (\sigma \hbn)\cdot \bv\,dS + \int_{\Omega^t} \tau \,:\,D\bv - p (\nabla \cdot \bv) - \left(\nabla \cdot \bu\right) q - \rho \mathbf{g} \cdot \bv \,dx dz \label{nonfunctwo}
\end{equation}
where $dS$ is the arclength element along $\partial\Omega^t$, and using the stress tensor notation for brevity.  Note that $\bu,\bv$ appear with at most first derivatives and $p,q$ appear without derivatives.  Next, because $\bv\in V_0$ satisfies $\bv=\bzero$ along the base and inflow surfaces, these parts of the integral over $\partial\Omega^t$ in \eqref{nonfunctwo} are zero.  The stress-free condition \eqref{topbc} on the top surface and the outflow condition \eqref{outflowbc} on the outflow boundary, which we denote $\partial_O \Omega^t$, now eliminate the unknowns from the boundary integral.  Thus we have the following formula for $\tilde F$:
\begin{align}
\tilde F &= \int_{\Omega^t} B_n |D\bu|^{\frac{1}{n} - 1} D\bu\,:\,D\bv - p (\nabla \cdot \bv) - \left(\nabla \cdot \bu\right) q \,dx dz \label{definetildeF} \\
    &\qquad  - \int_{\Omega^t} \rho \mathbf{g} \cdot \bv \,dx dz + \int_{\partial_O \Omega^t} \rho (H_{\text{out}}-z) \bg \cdot \bv \,dS \notag
\end{align}
We say $\bu\in V_D$ and $p\in Q$ solve the weak formulation if
\begin{equation}
\tilde F(\bu,p;\bv,q) = 0 \qquad \text{ for all } \bv\in V_0 \text{ and } q\in Q.  \label{weak}
\end{equation}
An equivalent formulation is proven in \cite[Theorem 3.8]{JouvetRappaz2011} to be well-posed under reasonable assumptions about the domain $\Omega$ and the boundary data.

The last two integrals in \eqref{definetildeF} can be regarded as source terms.  In fact, if the inflow velocity is zero and these source terms are also zero, for instance because gravity is turned off ($\bg=\bzero$), then the unique solution of \eqref{weak} is $\bu=\bzero$ and $p=0$.


\section{Implicit time-discretization and the icy domain update} \label{sec:implicitstep}

The main purpose of this paper is to propose an implicit domain-updating scheme to simultaneously solve the Glen-Stokes equations and the surface kinematical equation.  In this scheme we compute a greatly-simplified surrogate of the ice \emph{strain} (i.e.~deformation), namely a vertical-only displacement, defined over the time step and within the icy domain.  This displacement field is found simultaneously with the velocity and pressure by using the combined principles of mass and momentum conservation.  At each time step the current domain is, essentially, the reference configuration from which we compute the updated configuration \cite{GreveBlatter2009}.  Note that this scheme is stated using a time-discretized but continuous-space weak formulation.  In writing-out the equations of this scheme a significant task is to change variables in the integrals which define the Glen-Stokes weak form.  The corresponding finite element equations, considered in the next section, turn the domain update into a vertical-only \emph{mesh} displacement computed simultaneously with the discrete velocity and pressure fields.

We will describe the time-stepping using a backward Euler scheme.  Of course, better implicit schemes exist, and so in later work we will restate the problem as a differential algebraic equation (DAE) and apply suitable higher-order schemes.  Specifically, we will consider the second-order backward-differentiation formula (BDF2) \cite{AscherPetzold1998} and solver tools based on PETSc's \cite{Balayetal2020,BuelerBook} time-stepping TS object.
% possibly \texttt{firedrake-ts}; https://github.com/IvanYashchuk/firedrake-ts

Let $t_{n-1}$ and $t_n$ be consecutive times with step $\Delta t = t_n - t_{n-1} > 0$.  Suppose the model's state, namely the (current) ice geometry, is known at time $t_{n-1}$.  Denote this ice-filled domain as $\Lambda = \Omega^{t_{n-1}} \subset \RR^2$.  We will approximate the updated (new) domain $\Omega^n = \Omega^{t_n} \subset \RR^2$ using the weak form \eqref{weak} of the Glen-Stokes equations, a weak form of the surface kinematical equation \eqref{surfacekinematical} (see below), and input data for the surface mass balance $a(x,z,t)$.  Note that assumption \eqref{intervalassume} needs to hold for $\Lambda$, but, once we have solved all the upcoming equations, it must also hold for $\Omega^n$ so that time-stepping may proceed.

The coordinates on the current domain $\Lambda$ are denoted $(r,s)$.  We update the region by computing a change of coordinates $(r,s) \mapsto (x,z)$,
\begin{equation}
x(r,s)=r, \quad z(r,s)=s+c(r,s), \label{changecoords}
\end{equation}
so the new domain $\Omega^n$ is an image of $\Lambda$:
\begin{equation}
\Omega^n = \{(x,z)=(r,s+c(r,s)) \,\big|\, (r,s) \in \Lambda\}.  \label{updateddomain}
\end{equation}
(Note that the horizontal coordinate is not changed ($x=r$) but the $z$-coordinate is nontrivial.)  Here $c(r,s)$ is a scalar function which will be computed by solving coupled PDEs on $\Lambda$, thus the coupled solution determines $\Omega^n$ plus the ice velocity and pressure.  This $\Lambda \stackrel{\Delta t}{\to} \Omega^n$ domain update is sketched in Figure \ref{fig:domainupdate}.

\begin{figure}[h]

FIXME

\caption{The time $t_{n-1}$ (``current'') domain $\Lambda$ is the reference configuration for the time $t_n$ (``new'') domain $\Omega^n$.}
\label{fig:domainupdate}
\end{figure}

Let $\eta(r)$ denote the surface elevation of $\Lambda$.  Given \eqref{updateddomain}, the surface elevation for the new region is
\begin{equation}
h(r) = \eta(r) + c(r,\eta(r)).  \label{newsurfaceelevation}
\end{equation}
Note this is defined for $r$ in $\Lambda$ (i.e.~not $x$ in $\Omega^n$).  The value $c(r,\eta(r))$ is defined in a trace sense \cite{Evans2010}; see below for the Sobolev space which contains $c$.

FIXME: EDIT SWEEP UP TO HERE

The change of coordinates will be differentiable within $\Lambda$.  (This is because the nontrivial part, the function $c$, will solve an elliptic PDE; see below.)  From \eqref{changecoords} the Jacobian of the change is
\begin{equation}
J = \begin{pmatrix} \partial x / \partial r & \partial x / \partial s \\ {\large\strut} \partial z / \partial r & \partial z / \partial s \end{pmatrix} = \begin{pmatrix} 1 & 0 \\ \partial c/\partial r & 1+(\partial c/\partial s) \end{pmatrix}. \label{jacchange}
\end{equation}
For a generic, smooth, scalar-valued function $\tilde f(x,z)$ defined on $\Omega^n$ we define a new function on $\Lambda$,
    $$f(r,s) = \tilde f(r,s+c(r,s)),$$
and partial derivatives transform as
\begin{equation}
\begin{bmatrix} \partial \tilde f / \partial x \\ {\large\strut} \partial \tilde f / \partial z\end{bmatrix} = (J^\top)^{-1} \begin{bmatrix} \partial f / \partial r \\ {\large\strut} \partial f / \partial s\end{bmatrix} = \begin{pmatrix} 1 & \ell \\ 0 & k \end{pmatrix} \begin{bmatrix} \partial f / \partial r \\ {\large\strut} \partial f / \partial s\end{bmatrix} \label{changederivatives}
\end{equation}
where
\begin{equation}
j(r,s) = 1+\frac{\partial c}{\partial s}, \qquad k(r,s) = j(r,s)^{-1}, \qquad \ell(r,s) = - \frac{\partial c}{\partial r}(r,s) j(r,s)^{-1}. \label{definejkl}
\end{equation}

Note that $j(r,s)$ is the Jacobian determinant.  For stability we require that the change of coordinates not cause a local flip in orientation, though degeneration is allowed.  That is, we will require $j\ge 0$ or equivalently
\begin{equation}
\frac{\partial c}{\partial s} \ge -1. \label{differentialVI}
\end{equation}
(We reconsider the meaning of this restriction in \ref{sec:inequalities}.)

The time derivative $h_t$ in surface kinematical equation \eqref{surfacekinematical} will be approximated by $(h^n(r) - h(r))/\Delta t$.  Let
\begin{equation}
a^n(r) = a\left(r,\eta(r) + c(r,\eta(r)),t_n\right) \label{massbalance}
\end{equation}
be the mass balance computed at the updated surface location and the updated time.  Using \eqref{newsurfaceelevation}, our $O(\Delta t)$ approximation of \eqref{surfacekinematical}, along the top surface $s=\eta(r)$ of $\Lambda$ is
\begin{equation}
\frac{c(r,\eta(r))}{\Delta t} = a^n(r) - u(r,\eta(r))\,\eta'(r) + w(r,\eta(r)). \label{surfaceimplicit}
\end{equation}
Equation \eqref{surfaceimplicit} is a boundary condition at each top location $(r,\eta(r))$ on $\Lambda$.  In fact $\bu(r,s)=\left<u(r,s),w(r,s)\right>$ will also solve the Stokes equations (see below) on $\Lambda$, so condition \eqref{surfaceimplicit} couples the determination of the domain $\Omega^n$ with velocity and pressure.  Condition \eqref{surfaceimplicit} also couples surface elevation and mass balance during the solve, but a simplification is to use the time-splitting approximation $a^n \approx a(r,\eta(r),t_n)$ instead of \eqref{massbalance}.

We propose that the vertical displacement function $c(r,s)$, defined on the current domain $\Lambda$, solves a PDE problem, part of which is the boundary condition \eqref{surfaceimplicit}.  In strong form, $c$ will solve the Laplace equation with these boundary conditions:
\begin{align}
        \grad^2 c &= 0 &&\text{in } \Lambda \label{claplace} \\
                c &= \Delta t\,\left(a^n - u \eta' + w\right) &&\emph{top} \notag \\
                c &= 0 &&\text{\emph{base} \& \emph{inflow}} \notag \\
\grad c\cdot \hbn &= 0 &&\emph{outflow} \notag
\end{align}
Dirichlet conditions apply on most of the boundary, but a Neumann condition applies on the outflow (if present).  Assuming that $\Lambda$ is a well-behaved domain, this boundary-value problem is known to be well-posed in the \emph{decoupled} context where $a^n,u,w$ are given functions (with mild regularity assumptions) \cite{Evans2010}, but we intend to solve it in the coupled context where the Dirichlet condition on the top actually relates unknowns $c,u,w$.

The weak form of problem \eqref{claplace} can be stated as follows.  Noting that the outflow has a natural condition, we define the Laplacian bilinear form
\begin{equation}
a(c;e) = \int_\Lambda \frac{\partial c}{\partial r} \frac{\partial e}{\partial r} + \frac{\partial c}{\partial s} \frac{\partial e}{\partial s}\,dr ds  \label{asurface}
\end{equation}
and say that $c$ solves the (decoupled) weak form of \eqref{claplace} if $c \in W_D^{1,2}(\Lambda)$ and $e \in W_0^{1,2}(\Lambda)$ solves $a(c;e)=0$ for all $e \in W_0^{1,2}(\Lambda)$.  Note $c$ satisfies the nonhomogeneous Dirichlet conditions on the top [FIXME: HOW GIVEN \texttt{EquationBC}?], while $e$ is zero on the top, and both are zero on the base and inflow.

FIXME a fully-implicit version of \eqref{surfaceimplicit} would be
\begin{equation}
\frac{c(r,\eta(r))}{\Delta t} = a^n(r) - u(r,\eta(r))\,\frac{d}{dr}(h(r)) + w(r,\eta(r)). \label{surfaceimplicitfully}
\end{equation}
unfortunately this is more difficult because it involves an ``oblique derivative'' condition on \eqref{claplace}; but it can be tried out fairly easily with firedrakes \texttt{EquationBC}


\section{Weak form of the coupled equations} \label{sec:weakformcoupled}

The change of variables defined in the last section will rewrite the weak form integrals.  For example, consider a generic scalar-valued $L^1$ function $\tilde f(x,z)$ defined on $\Omega^n$.  By the change of variables theorem we have
\begin{equation}
\int_{\Omega^n} \tilde f(x,z)\,dx dz = \int_\Lambda f(r,s) \, j(r,s)\,dr ds, \label{changeintegral}
\end{equation}
with weight $j(r,s)$ defined in \eqref{definejkl} and $f(r,s) = \tilde f(r,s+c(r,s))$.  Also note the following expansions of velocity derivatives (strain rates) with respect to $(x,z)$ which follow from \eqref{changederivatives} and \eqref{definejkl}.  Also, we use component notation $\bu = \left<u_0,u_2\right>$, compatible with coordinate notation $(x,z)$ on $\Omega$:
\begin{align*}
\grad \cdot \tilde \bu &= \frac{\partial u_0}{\partial r} + \ell \frac{\partial u_0}{\partial s} + k \frac{\partial u_2}{\partial s} \\
D \tilde \bu &= \begin{pmatrix} \partial u_0/\partial r + \ell (\partial u_0/\partial s) & \gamma \\
  {\large\strut} \gamma & k (\partial u_2/\partial s)\end{pmatrix} \\
|D \tilde \bu|^2 &= \frac{1}{2}\left(\frac{\partial u_0}{\partial r} + \ell \frac{\partial u_0}{\partial s}\right)^2 + \gamma^2 + \frac{1}{2}\left(k \frac{\partial u_2}{\partial s}\right)^2
\end{align*}
where
    $$\gamma = \frac{1}{2} \left(k \frac{\partial u_0}{\partial s} + \frac{\partial u_2}{\partial r} + \ell \frac{\partial u_2}{\partial s}\right)$$
is a notational simplification only.

FIXME state the weak form using independent variables $r,s$ and integrals over $\Lambda$; equations are for solution functions $\bu,p,c$ with test functions $\bv,q,e$

From weak forms \eqref{definetildeF} and \eqref{asurface}, the proposed [DRAFT] weak form is
\begin{align}
F(\bu,p,c;\bv,q,e) &= \int_\Lambda B_n |D\bu|^{\frac{1}{n} - 1} D\bu\,:\,D\bv\, j\,dr ds [FIXME] \label{defineF} \\
    &\quad  - \int_\Lambda p \left(j \frac{\partial v_0}{\partial r} - \frac{\partial c}{\partial r} \frac{\partial v_0}{\partial s} + \frac{\partial v_2}{\partial s}\right) \,dr ds \notag \\
    &\quad - \int_\Lambda q \left(j \frac{\partial u_0}{\partial r} - \frac{\partial c}{\partial r} \frac{\partial u_0}{\partial s} + \frac{\partial u_2}{\partial s}\right)\,dr ds \notag \\
    &\quad  - \int_\Lambda \rho \mathbf{g} \cdot \bv \, j\,dr ds \notag \\
    &\quad + \int_{\partial_O \Lambda} \rho (H_{\text{out}}-s-c) \bg \cdot \bv \,dS \notag \\
    &\quad + \int_\Lambda \frac{\partial c}{\partial r} \frac{\partial e}{\partial r} + \frac{\partial c}{\partial s} \frac{\partial e}{\partial s}\,dr ds \notag
\end{align}
FIXME in above: $|D\bu|$ and $D\bu:D\bv$ need expansion; how to regularize that term; how to write weak form for $c$ when we are actually going to use \texttt{EquationBC}?

The solution of the weak form is a list of three functions, $\bu = \left<u_0,u_2\right> \in V_D$, $p\in L^{p'}(\Lambda)$, and $c\in W^{1,2}_D(\Lambda)$, such that
\begin{equation}
F(\bu,p,c;\bv,q,e) = 0 \text{ for all } \bv = \left<v_0,v_2\right> \in V_0, \, q\in L^{p'}(\Lambda), \text{ and } e \in W^{1,2}_0(\Lambda).
\end{equation}
All trial and test functions are functions of $r$ and $s$ in $\Lambda$.

FIXME result is following block structure


\section{Inequality constraints and mass accounting} \label{sec:inequalities}

\begin{lemma}
Assume that the ice thickness is well defined at time $t_{n-1}$, that is, assume
\eqref{intervalassume}.  If inequality \eqref{differentialVI} holds then the updated ice surface elevation
    $$h(r) = \sup_{s\in I(r)}\{z(r,s)\}$$
satisfies \eqref{admissibility}, i.e.~$h(r)\ge b(r)$.
\end{lemma}

\begin{proof}
FIXME? This is the fundamental theorem of calculus:
    $$h(r) - b(r) = \int_{I(r)} 1\,ds \le \int_{I(r)} - \frac{\partial c}{\partial s}\,ds$$
\end{proof}


\section{Finite element approximation}  \label{sec:finiteelement}

FIXME: numerical solver should check element orientation under change of coordinates (above); if the solver flips an element then it is bad; also check new element aspect ratio and (presumably) remesh if that is bad; the initial iterate for the (SNES-based) solver is clear: $\bu,p$ come from solution of previous time step, and $b$ starts at zero



\small

\bigskip
\bibliography{simp}
\bibliographystyle{siam}

\end{document}
